%%%%%\usepackage{amscls}
\usepackage{amsfonts}
%%%%%\usepackage[brazil]{babel} %linguagem do documento
%%%%%\usepackage{babel}
%%%%%\usepackage[utf8]{inputenc} %reconhece acento e cedilha
\usepackage{amsmath}
\usepackage{amssymb}
%%%%%\usepackage{amsthm}
\usepackage{array}
\usepackage[portuguese]{babel}
\usepackage{babelbib}
\usepackage{bm}
\usepackage{booktabs}
\usepackage{boxedminipage}
\usepackage{caption}
%%%%%\usepackage{cancel}
\usepackage{changepage}
\usepackage{cite}
\usepackage[usenames,dvipsnames,svgnames,table]{xcolor} %\usepackage[usenames]{color} %permite letras coloridas
\usepackage{easylist}
\usepackage{esint}
\usepackage{eucal}
\usepackage{fancyhdr}
\usepackage{float}
\usepackage[T1]{fontenc}
%%%%%\usepackage{fullpage}
%%%%%\usepackage{geometry}
\usepackage[top=2cm,left=1.5cm,right=1.5cm,bottom=1.5cm]{geometry} %margens
%%%%%\usepackage[margin=1in, paperwidth=8.5in, paperheight=11in]{geometry}
%%%%%\usepackage[top=1in, bottom=1in, left=1in, right=1in]{geometry}
%%%%%\usepackage{glossaries}
\usepackage{graphicx} %permite inserir figuras
\usepackage{hyperref}
\usepackage{indentfirst}
\usepackage{inputenc}
%\usepackage{itemize}
\usepackage{latexsym}
\usepackage{listings}
\usepackage{makeidx} %pra criar índice remissivo
\usepackage{mathptmx}
\usepackage{mathrsfs} %permite o uso de letras trabalhadas
%%%%%\usepackage{mathtools}
%%%%%\usepackage[fleqn]{mathtools}
\usepackage[version=3]{mhchem}
\usepackage{microtype}
\usepackage{multicol}
%%%%%\usepackage{named}
\usepackage[normalem]{ulem} %permite sublinhar palavras
%%%%%\usepackage{natbib}
\usepackage{paralist}
%\usepackage{pxfonts} %permite simbolos matemáticos
\usepackage{mathrsfs} %permite uso de fontes para conjuntos
%%%%%\usepackage{pdfpages}
%%%%%\usepackage{pgfplots}
%%%%%\usepackage{pifont}
\usepackage{rotating}
\usepackage{setspace}
%%%%%\usepackage{showkeys} %for troubleshooting \label \ref
%%%%%\usepackage{showidx} %for troubleshooting index
\usepackage{subfiles}
\usepackage{subcaption}
\usepackage{syntonly} %speedup work desabling pdf converse \syntaxonly
\usepackage{textcomp}
\usepackage{theorem}
%%%%%\usepackage{todonotes}
%%%%%\usepackage{siunitx}
%%%%%\usepackage{ulem}
\usepackage{url}
%\usepackage[usenames]{color} %permite letras coloridas
\usepackage{verbatim}
\usepackage{wrapfig}
%%%%\usepackage{xypic}
%%%%%%%%%%%%%%%%%%%%%%%%%%%%%%%%%%%%%%%%%%%%%%%%%%%%%%%%%%%%%%%%%%%%%%%%%%%%%%%%%%%%%%%%%%%%%%%%%%
\usepackage{enumerate}
%%%%%%%%%%%%%%%%%%%%
%\begin{comment}
\renewcommand{\labelitemi}{$\bullet$}
\renewcommand{\labelitemii}{$\cdot$}
\renewcommand{\labelitemiii}{$\diamond$}
\renewcommand{\labelitemiv}{$\ast$}
%\end{comment}
%%%%%%%%%%%%%%%%%%%%%%%%%%%%%%%%%%%%%%%%%%%%%%%%%%%%%%%%%%%%%%%%%%%%%%%%%%%%%%%%%%%%%%%%%%%%%%%%%%
\usepackage{enumitem}
%\begin{comment}
\setlistdepth{12}
\newlist{enumitem}{enumerate}{12}
\setlist[enumitem,1]{label=\roman*)}
\setlist[enumitem,2]{label=\alph*)}
\setlist[enumitem,3]{label=\arabic*)}
\setlist[enumitem,4]{label=(\roman*)}
\setlist[enumitem,5]{label=(\alph*)}
\setlist[enumitem,6]{label=(\arabic*)}
\setlist[enumitem,7]{label=\roman*)}
\setlist[enumitem,8]{label=\alph*)}
\setlist[enumitem,9]{label=\arabic*)}
\setlist[enumitem,10]{label=(\roman*)}
\setlist[enumitem,11]{label=(\alph*)}
\setlist[enumitem,12]{label=(\arabic*)}
%\end{comment}
%%%%%%%%%%%%%%%%%%%%%%%%%%%%%%%%%%%%%%%%%%%%%%%%%%%%%%%%%%%%%%%%%%%%%%%%%%%%%%%%%%%%%%%%%%%%%%%%%%
\usepackage{tikz}
\usetikzlibrary{matrix,shapes.geometric,arrows,trees,positioning,calc}
\begin{comment}
%%%%%%%%%%%%%START DEFINITIONS%%%%%%%%%%%%%
% types of possible settings
\tikzstyle{RECTANGLE_2} = [rectangle, draw, text width=5em, text centered, rounded corners, minimum height=4em]
\tikzstyle{RECTANGLE_3} = [rectangle, rounded corners, minimum width=3cm, minimum height=1cm,text centered, draw=black, fill=red!80]
\tikzstyle{RECTANGLE_4} = [rectangle, draw, fill=blue!20, text width=3cm, text centered, minimum height=4em]
\tikzstyle{RECTANGLE_5} = [rectangle, minimum width=3cm, minimum height=1cm, text centered, text width=3cm]
\tikzstyle{RECTANGLE_6} = [rectangle, draw, fill=blue!20, text width=5em, text centered, rounded corners, minimum height=4em]
\tikzstyle{RECTANGLE_7} = [rectangle, draw, fill=blue!20, text width=5em, text centered, rounded corners, minimum height=4em]
\tikzstyle{RECTANGLE_8} = [rectangle, draw, align=left, fill=blue!20]
\tikzstyle{RECTANGLE_1} = [rectangle, rounded corners, minimum width=1cm, minimum height=1cm,text centered, draw=black, fill=green!30]
\tikzstyle{DIAMOND_1} = [diamond, draw, fill=blue!20, text width=4.5em, text badly centered, node distance=4cm, inner sep=0pt]
\tikzstyle{DIAMOND_2} = [diamond, minimum width=3cm, minimum height=1cm, text centered, draw=black, fill=green!30]
\tikzstyle{DIAMOND_3} = [diamond, draw, text width=4.5em, text badly centered, node distance=3cm, inner sep=0pt]
\tikzstyle{DIAMOND_4} = [diamond, draw, fill=blue!20, text width=4.5em, text badly centered, node distance=3cm, inner sep=0pt]
\tikzstyle{DIAMOND_5} = [diamond, draw, fill=blue!20, text width=4.5em, text badly centered, node distance=3cm, inner sep=0pt]
\tikzstyle{DIAMOND_6} = [diamond, draw, fill=blue!20, text width=4.5em, text badly centered, node distance=4cm, inner sep=0pt]
\tikzstyle{DIAMOND_7} = [diamond, draw, align=left, fill=blue!20]
\tikzstyle{ELLIPSE_1} = [draw, ellipse,fill=red!20, node distance=3cm, minimum height=2em]
\tikzstyle{ELLIPSE_2} = [draw, ellipse,fill=red!20, node distance=3cm, minimum height=2em]
\tikzstyle{ELLIPSE} = [draw, ellipse,fill=red!20, node distance=3cm, minimum height=2em]
\tikzstyle{TRAPEZIUM_1} = [trapezium,trapezium left angle=70,trapezium right angle=-70,minimum height=0.6cm, draw, fill=blue!20, text width=4.5em, text badly centered, node distance=3cm, inner sep=0pt]
\tikzstyle{TRAPEZIUM_2} = [trapezium, trapezium left angle=70, trapezium right angle=110, minimum width=3cm, minimum height=1cm, text centered, draw=black, fill=blue!30]
\tikzstyle{TRAPEZIUM_3} = [trapezium,trapezium left angle=70,trapezium right angle=-70,minimum height=0.6cm, draw, fill=blue!20, text width=4.5em, text badly centered, node distance=3cm, inner sep=0pt]
\tikzstyle{ARROW} = [thick,->,>=stealth]
\tikzstyle{LINE} = [draw, -latex']
\tikzstyle{MYLINE} = [draw, ->,  thick, shorten <=4pt, shorten >=4pt]
\tikzstyle{TEXT_1}=[draw,text centered,minimum size=6em,text width=5.25cm,text height=0.34cm]
\tikzstyle{TEXT_2}=[draw,text centered,minimum size=2em,text width=2.75cm,text height=0.34cm]
\tikzstyle{TEXT_3}=[draw,minimum size=2.5em,text centered,text width=3.5cm]
\tikzstyle{TEXT_4}=[draw,minimum size=3em,text centered,text width=6.cm]
\tikzstyle{CIRCLE_1}=[draw,shape=circle,inner sep=2pt,text centered, node distance=3.5cm]
\tikzstyle{CIRCLE_2}=[draw,shape=circle,inner sep=4pt,text centered, node distance=3.cm]
%%%%%%%%%%END DEFINITIONS%%%%%%%%%%
\end{comment}
%%%%%%%%%%%%%%%%%%%%%%%%%%%%%%%%%%%%%%%%%%%%%%%%%%%%%%%%%%%%%%%%%%%%%%%%%%%%%%%%%%%%%%%%%%%%%%%%%%
%alguns pacotes nao sao reconhecidos, ter atencao quais usar em differents computadores, tambeem alguns pacotes entram em conflito.
\newtheorem{theorem}{Theorem}
\newtheorem{lemma}{Lemma}
\newtheorem{definition}{Defini\c{c}\~{a}o}
\newtheorem{notation}{Notation}
%%%%%%%%%%%%%%%%%%%%%%%%%%%%%%%%%%%%%%%%%%%%%%%%%%%%%%%%%%%%%%%%%%%%%%%%%%%%%%%%%%%%%%%%%%%%%%%%%%%
\bibliographystyle{babplain}
\makeindex
